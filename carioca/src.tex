\documentclass[12pt, a4paper]{article}
\usepackage[utf8]{inputenc}
\usepackage[T1]{fontenc}
\usepackage{ebgaramond} % Fonte Clássica
\usepackage{geometry}
\usepackage{setspace}
\usepackage{ifthen}
\usepackage{tocloft} % Pacote para personalizar o índice


% Margens
\geometry{top=2.5cm, bottom=2.5cm, left=3cm, right=3cm}

% Remove numeração e indentação
\setlength{\parindent}{0pt}

% --- configurações do índice ---
\renewcommand{\contentsname}{}
\renewcommand{\cftdotsep}{\cftnodots}
\tocloftpagestyle{empty}

% --- comando para letra inicial ---
\newcommand{\letra}[1]{
  \newpage
  \addcontentsline{toc}{section}{#1}
  \vspace*{2cm}
  \raggedleft
  \textbf{\fontsize{72}{80}\selectfont #1}
  \par\vspace{1cm}
}


% --- COMANDO AJUSTADO ---
\newcommand{\entry}[5]{%
    \addcontentsline{toc}{subsection}{#1}
    \noindent
    \begin{minipage}{\textwidth} % Garante que o verbete não parte a meio
      \raggedright 
      % Título Gigante
      {\fontsize{30}{38}\selectfont \textbf{#1}\par}

      \vspace{0.7cm} % MAIOR ESPAÇAMENTO (Título -> Pronúncia)

      {\large [ #1 ] \textbf{\textit{#2}}  -  #3 \par}
      \rule{\textwidth}{0.4pt}
      {\large \setstretch{1.1} #4\\ #5 \par}
    \end{minipage}
    \vspace{1.5cm} % Espaço entre os diferentes verbetes
}

\begin{document}

\begin{titlepage}
    \centering % Centraliza tudo horizontalmente
    \vspace*{\fill} % Empurra para o centro vertical
    
    % 1. Conceito (Título)
    {\fontsize{70}{80}\selectfont \textbf{dicionário}\par}
    \vspace{0.2cm}
    {\fontsize{40}{50}\selectfont \textbf{para um Tuga}\par}
    
    \vspace{1.5cm}
    
    % 2. [ pronuncia ] tipo • Língua
    {\Large [ pronuncia ] \textbf{\textit{classe}}  -  Carioca\par}
    \vspace{0.5cm}
    
    % 3. Linha Horizontal
    \rule{0.8\textwidth}{0.5pt}
    
    \vspace{0.8cm}
    
    % 4. Definição do Dicionário
    \begin{minipage}{0.8\textwidth}
        \centering \Large \setstretch{1.3}
        compilação essencial de gírias, expressões e modos de falar do rio de janeiro, traduzidas e explicadas para o entendimento do tuga.
    \end{minipage}
    
    \vspace*{\fill} % Empurra para o centro vertical
\end{titlepage}


% --- LISTA DE PALAVRAS ---
\pagestyle{empty} %
\tableofcontents
\newpage

\pagestyle{plain} % Ativa a numeração no rodapé
\setcounter{page}{0}

\letra{0}
\entry{0800}{adjetivo}{Carioca}
{gratuito; de borla; que não custa dinheiro.}
{\textbf{ex.} o concerto será 0800.}

\letra{a}
\entry{a porra toda}{subst. composto}{Carioca}
{tudo; a totalidade das coisas, com eventual conotação agressiva. 'partimos a porra toda'.}

\entry{aê}{interjeição}{Carioca}
{partícula para iniciar uma frase; advérbio de lugar: ali; por lá.}

\entry{arroz}{substantivo}{Carioca}
{pessoa que apenas acompanha; indivíduo que vive rodeado de atrações mas nunca se envolve. sinónimos: arame-liso, mestre-sala.}

\letra{b}
\entry{beleza}{substantivo}{Carioca}
{cumprimento: 'tudo bem?'; aceitação: 'está bem'; exaltação: 'que espetáculo!'.}

\entry{boiola}{substantivo}{Carioca}
{homossexual masculino; paneleiro; maricas.}

\entry{bolado}{adjetivo}{Carioca}
{preocupado; confuso; com incompreensão momentânea.}

\entry{brotar}{verbo}{Carioca}
{aparecer num lugar; chegar de repente.}

\entry{bucha}{substantivo}{Carioca}
{pessoa que se arma em esperta ('malandro'), mas falha; 'prego'.}

\letra{c}
\entry{cabaço}{adjetivo}{Carioca}
{pessoa trapalhona; que faz asneiras.}

\entry{caído}{adjetivo}{Carioca}
{evento pouco divertido ou desanimado; pessoa pouco atraente.}

\entry{caô}{substantivo}{Carioca}
{mentira; treta; peta; história mal contada.}

\entry{caraca}{interjeição}{Carioca}
{expressão de surpresa ou espanto; versão suave de asneira.}

\entry{coé}{interjeição}{Carioca}
{aglutinação de 'qual é'; cumprimento, confrontação ou dúvida.}

\entry{conto}{substantivo}{Carioca}
{unidade monetária (dinheiro); ex: 10 conto (10 reais/euros).}

\entry{cria}{substantivo}{Carioca}
{pessoa autêntica; nascida e criada na comunidade; 'da casa'.}

\letra{d}
\entry{dá uma moral}{expressão}{Carioca}
{pedir auxílio ou ajuda; negociar uma pequena vantagem.}

\entry{dar o papo}{expressão}{Carioca}
{falar a verdade; ser direto; avisar sobre algo importante.}

\letra{f}
\entry{filhadaputa}{interj. / adjetivo}{Carioca}
{expressão de descontentamento; adjetivo usado para ofender.}

\entry{fluir}{verbo}{Carioca}
{dar certo; correr bem.}

\entry{foda}{adjetivo}{Carioca}
{difícil ('lixado'); muito bom ('brutal'); impressionante.}

\entry{fura-olho}{substantivo}{Carioca}
{indivíduo que usufrui das glórias alheias; traidor em relações.}

\letra{g}
\entry{gastar}{verbo}{Carioca}
{gozar com a cara de alguém; brincar.}

\entry{geral}{substantivo}{Carioca}
{toda a gente; a malta toda.}

\entry{goxtosa}{adjetivo}{Carioca}
{mulher de formas físicas harmoniosas; 'boazona'.}

\letra{i}
\entry{irado}{adjetivo}{Carioca}
{espetacular; muito bom; brutal.}

\letra{j}
\entry{já é!}{interjeição}{Carioca}
{concordância; sinónimo: siga, bora.}

\letra{m}
\entry{maluco}{substantivo}{Carioca}
{gajo; sujeito; indivíduo.}

\entry{maneiro}{adjetivo}{Carioca}
{muito fixe; espetacular.}

\entry{mermão}{substantivo}{Carioca}
{aglutinação de 'meu irmão'; usado para dirigir-se a qualquer pessoa.}

\entry{meter o pé}{expressão}{Carioca}
{ir embora rapidamente; bazar.}

\entry{mó}{advérbio}{Carioca}
{aglutinação de 'maior'; muito. ex: 'mó otário' (grande otário).}

\entry{muleque}{substantivo}{Carioca}
{rapaz; miúdo; amigo (usado para homens adultos).}

\letra{n}
\entry{na mão do palhaço}{expressão}{Carioca}
{estado de pessoas entorpecidas; bêbado.}

\entry{nego}{pronome}{Carioca}
{substitui 'eles' ou 'alguém'; verbo concorda no singular.}

\entry{night}{substantivo}{Carioca}
{saída à noite; diversão noturna.}

\letra{p}
\entry{papo reto}{expressão}{Carioca}
{conversa séria; falar sem rodeios.}

\entry{parada}{substantivo}{Carioca}
{coisa genérica; 'cena'.}

\entry{paraíba}{substantivo}{Carioca}
{indivíduo do nordeste do brasil (pejorativo no rio); associado a 'saloio'.}

\entry{partiu}{interjeição}{Carioca}
{vamos? (pergunta); siga! (afirmação de início de ação).}

\entry{pega a visão}{expressão}{Carioca}
{toma nota; percebe isto; abre a pestana.}

\entry{peidão}{adjetivo}{Carioca}
{cobarde; frouxo.}

\entry{pela-saco}{substantivo}{Carioca}
{pessoa chata; graxista; lambe-botas.}

\entry{perdeu a linha}{expressão}{Carioca}
{cometeu um ato insensato; 'passou-se'.}

\entry{porra}{interjeição}{Carioca}
{espanto ou raiva; 'cena'; intensidade ('frio da porra').}

\entry{porrada}{substantivo}{Carioca}
{pancadaria; grande quantidade ('uma porrada de gente').}

\letra{s}
\entry{se liga}{expressão}{Carioca}
{apelo por atenção; 'está atento'.}

\entry{sinistro}{adjetivo}{Carioca}
{brutal; muito bom; difícil; perigoso.}

\letra{t}
\entry{tamo junto}{expressão}{Carioca}
{estamos juntos; conta comigo.}

\entry{tu}{pronome}{Carioca}
{tu; no rio, seguido por verbo na 3ª pessoa ('tu vai').}

\letra{v}
\entry{vacilão}{adjetivo}{Carioca}
{pessoa que falha; lento a perceber as coisas.}

\letra{x}
\entry{x-9}{substantivo}{Carioca}
{delator; bufo; chibo.}

\end{document}
