\documentclass[12pt, a4paper]{article}
\usepackage[utf8]{inputenc}
\usepackage[T1]{fontenc}
\usepackage{ebgaramond} % Fonte Clássica
\usepackage{geometry}
\usepackage{setspace}
\usepackage{ifthen}
\usepackage{tocloft} % Pacote para personalizar o índice


% Margens
\geometry{top=2.5cm, bottom=2.5cm, left=3cm, right=3cm}

% Remove numeração e indentação
\setlength{\parindent}{0pt}

% --- configurações do índice ---
\renewcommand{\contentsname}{}
\renewcommand{\cftdotsep}{\cftnodots}
\tocloftpagestyle{empty}

% --- comando para letra inicial ---
\newcommand{\letra}[1]{
  \newpage
  \addcontentsline{toc}{section}{#1}
  \vspace*{2cm}
  \raggedleft
  \textbf{\fontsize{72}{80}\selectfont #1}
  \par\vspace{1cm}
}


% --- COMANDO AJUSTADO ---
\newcommand{\entry}[5]{%
    \addcontentsline{toc}{subsection}{#1}
    \noindent
    \begin{minipage}{\textwidth} % Garante que o verbete não parte a meio
      \raggedright 
      % Título Gigante
      {\fontsize{30}{38}\selectfont \textbf{#1}\par}

      \vspace{0.7cm} % MAIOR ESPAÇAMENTO (Título -> Pronúncia)

      {\large [ #1 ] \textbf{\textit{#2}}  -  #3 \par}
      \rule{\textwidth}{0.4pt}
      {\large \setstretch{1.1} #4\\ #5 \par}
    \end{minipage}
    \vspace{1.5cm} % Espaço entre os diferentes verbetes
}

\begin{document}

\begin{titlepage}
  \centering 
  \vspace*{\fill} 

  {\fontsize{70}{80}\selectfont \textbf{dicionário}\par}
  \vspace{0.2cm}
  {\fontsize{40}{50}\selectfont \textbf{de um algarvie}\par}

  \vspace{1.5cm}

  {\Large [ al.gar.'vju ] \textbf{\textit{classe}} - \textbf{Algarvie}\par}

  \rule{0.8\textwidth}{0.5pt}
  \vspace{0.2cm}

  \begin{minipage}{0.8\textwidth}
    \centering \Large \setstretch{1.3}
    compilação de termos, dizeres e pronuncias típicas do algarve, recolhidas por vítor madeira. já com o 'nove acorde ortográfique'.
  \end{minipage}
  \vspace*{\fill} 
\end{titlepage}


% --- ÍNDICE ---
\pagestyle{empty} 
\tableofcontents
\thispagestyle{empty} 
\newpage

% --- INÍCIO DO CONTEÚDO ---
\pagestyle{plain} 
\setcounter{page}{0} 

% --- LISTA DE TERMOS ---

\letra{a}
\entry{abuscar}{verbo}{Algarvie}
{buscar; procurar.}
{\textbf{ex.} us cãs foram abuscar os coelhes no mê do mate}

\entry{acarditar}{verbo}{Algarvie}
{acreditar.}
{\textbf{ex.} moce, até parace que n'acarditas em mim}

\entry{acêfa}{substantivo}{Algarvie}
{ceifa. trabalho agrícola de corte de cereais.}
{}

\letra{b}
\entry{barreca}{substantivo}{Algarvie}
{barraca.}
{prontes, já tá a barreca armada}

\entry{batenêra}{substantivo}{Algarvie}
{betoneira; máquina de fazer betão.}
{\textbf{ex.} moss, liga a batenêra}

\entry{bnite}{adjetivo}
{bonito. pronúncia com elisão da vogal 'o'.}
{}

\entry{bucha}{substantivo}{Algarvie}
{almoço; merenda ou lanche.}
{\textbf{ex.} iste já tá na hora da bucha}

\entry{buftada}{substantivo}{Algarvie}
{chapada; bofetada.}
{\textbf{ex.} tás aqui, tás a levar uma buftada}

\letra{c}
\entry{cabele}{substantivo}{Algarvie}
{cabelo.}
{\textbf{ex.} cortê o cabele ontem}

\entry{cagade}{adjetivo}{Algarvie}
{sujo; sortudo (dependendo do contexto); falhado.}
{}

\entry{cagorre}{substantivo}{Algarvie}
{susto.}
{\textbf{ex.} aquele maldeçoade amandou-me um cagorre c'até vi luzes}

\entry{caguifa}{substantivo}{Algarvie}
{medo.}
{\textbf{ex.} de nôte tenh'uma caguifa, mas de dia na tenhe}

\entry{calcaire}{substantivo}{Algarvie}
{calcário. rocha sedimentar comum no algarve.}
{}

\entry{contrebute}{substantivo}{Algarvie}
{contributo.}
{\textbf{ex.} dê o mê contrebute prá festa}

\entry{cornes}{substantivo}{Algarvie}
{cornos; testa; cabeça.}
{\textbf{ex.} u cagade tem a mania de marcar gôles c'us cornes}

\letra{e}
\entry{espetácl}{substantivo}{Algarvie}
{espetáculo.}
{\textbf{ex.} foi um espetácl de festa}

\letra{f}
\entry{faz-avôr}{expressão}{Algarvie}
{se faz favor; por favor.}
{\textbf{ex.} cumadre, dêm'aí o guidal, faz-avôr}

\entry{fêjão carite}{substantivo}{Algarvie}
{feijão-frade.}
{\textbf{ex.} goste munte duma saladinha com fêjão carite}

\letra{l}
\entry{ladêra}{substantivo}{Algarvie}
{ladeira; descida acentuada.}
{\textbf{ex.} tem cuidade a descer a ladêra pra na caíres}

\entry{lagues}{topónimo}{Algarvie}
{lagos (cidade algarvia).}
{}

\entry{lariar a pevide}{expressão}{Algarvie}
{vadiar; passear sem fazer nada útil.}
{\textbf{ex.} o manel anda a lariar a pevide}

\entry{luzescús}{substantivo}{Algarvie}
{pirilampos.}
{\textbf{ex.} esta nôte tá tude chê de luzescús}

\letra{m}
\entry{macheia}{substantivo}{Algarvie}
{mão-cheia; grande quantidade.}
{\textbf{ex.} hoje vi uma macheia de combois}

\entry{melanças}{substantivo}{Algarvie}
{melancias (geralmente usado no plural).}
{\textbf{ex.} na tem aí adube prás melanças?}

\entry{miga}{substantivo}{Algarvie}
{abreviação de amiga/amigo em contexto muito familiar.}
{\textbf{ex.} miga, passa-me o pão}

\letra{o}
\entry{ora ora}{interjeição}{Algarvie}
{expressão de perplexidade ou indignação.}
{\textbf{ex.} ora ora, vejem bem este linde service!}

\letra{p}
\entry{panite}{substantivo}{Algarvie}
{um pão pequeno; papo-seco.}
{}

\entry{porrete}{substantivo}{Algarvie}
{copo de vinho, seguido de outro.}
{\textbf{ex.} bebeu um porrete e foi-se}

\entry{pu diéb}{interjeição}{Algarvie}
{expressão de que cheira mal (ao diabo).}
{\textbf{ex.} pu diéb, quem foi o maldeçoade?}

\entry{pugrama}{substantivo}{Algarvie}
{programa.}
{\textbf{ex.} viste o pugrama da festa?}

\letra{s}
\entry{seca-adegas}{adjetivo}{Algarvie}
{bêbado; borrachão; alguém que bebe o vinho todo.}
{}

\entry{service}{substantivo}{Algarvie}
{serviço.}
{\textbf{ex.} que belo service me saíste}

\entry{sopape}{substantivo}{Algarvie}
{sopapo; tabefe; chapada forte.}
{}

\end{document}
